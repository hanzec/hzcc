
\documentclass{article}
\usepackage{fullpage}
\usepackage{hyperref}
\usepackage{color}
\usepackage{listings}
\usepackage{longtable}

\definecolor{dkgreen}{rgb}{0,0.6,0}
\definecolor{muave}{rgb}{0.58,0,0.82}
\definecolor{dkred}{rgb}{0.6,0,0}
\definecolor{dkblue}{rgb}{0,0,0.7}

\lstset{frame=none,
  language=C,
  % aboveskip=3mm,
  % belowskip=3mm,
  xleftmargin=2em,
  % xrightmargin=2em,
  showstringspaces=false,
  columns=flexible,
  basicstyle={\ttfamily},
  numbers=left,
  keywordstyle=\color{dkblue},
  directivestyle=\color{dkred},
  commentstyle=\color{dkgreen},
  stringstyle=\color{muave},
  breaklines=true,
  breakatwhitespace=true,
  tabsize=2
}

\lstdefinestyle{jvm}{
  % aboveskip=3mm,
  % belowskip=3mm,
  xleftmargin=2em,
  % xrightmargin=2em,
  showstringspaces=false,
  columns=flexible,
  basicstyle={\ttfamily},
  numbers=left,
  moredelim=[s][\color{black}]{Ljava}{;},
  morecomment=[l][\color{dkgreen}]{;},
  morecomment=[l][\color{magenta}]{;;},
  keywords={class,public,static,super,method,code,end},
  keywordstyle=\color{dkblue},
  breaklines=true,
  breakatwhitespace=true,
  tabsize=8
}

\newcommand{\parser}{3}
\newcommand{\typecheck}{4}
\newcommand{\codegen}{5}
\newcommand{\flowgen}{6}

\newcommand{\gradeline}{ \cline{1-2} \cline{4-4} ~\\[-1.5ex] }

\newenvironment{gradetable}{\begin{longtable}{@{~~}rrcp{5in}} \multicolumn{2}{l}{\bf Points} & & {\bf Description}\\ \gradeline}{\end{longtable}}

\newcommand{\mainitem}[2]{\pagebreak[2] {\bf #1} &&& {\bf #2}}
\newcommand{\mainpara}[1]{~ &&& {#1} }
\newcommand{\inneritem}[2]{~ & #1 && #2}
\newcommand{\innerpara}[1]{~ & ~ && #1}

\newcommand{\markedinneritem}[2]{\multicolumn{1}{@{}l}{$\dagger$} & #1 && #2}


\title{COM S 440/540 Project part \flowgen}
\author{Code generation: control flow}
\date{}

\newenvironment{quotewide}[1]{\begin{minipage}[b]{#1} \vspace{1mm} \begin{quote}} {\end{quote} \vspace{1mm} \end{minipage}}

\begin{document}

\maketitle

%======================================================================
\section{Requirements for part \flowgen{}}
%======================================================================

When executed with a mode of \flowgen{},
your compiler should read the specified input file,
and check it for correctness (including type checking) as done in part~\typecheck.
If there are no errors, then your compiler should output
an equivalent program in our target language
(still Java assembly).
For this part of the project,
your compiler must generate correct code for expressions (from part~\codegen),
and for (possibly nested) branching statements and loops.
As usual,
error messages should be written to standard error,
and your compiler may make a ``best effort'' to continue processing
the input file, or exit.
Specifically, your part~\codegen{} solution should be modified to include
the following.
\begin{itemize}
  \item Type checking of conditions (expressions) within
        if statements, while loops, do-while loops, and
        for loops: the condition expression should be a numeric type
        (char, int, or float).

  \item Code generation for
        if statements (with and without else),
        while loops,
        do-while loops,
        and for loops.

  \item Code generation for break and continue statements;
        error messages when they are not within a loop.

  \item The Java {\tt main()} function should \emph{not}
        display the return value of the C {\tt main()} function.
\end{itemize}

%======================================================================
\section{Checking your generated code} \label{SEC:checking}
%======================================================================

Ultimately,
you should be able to assemble the code generated by your compiler
(using the Krakatau assembler)
to obtain a class file.
You can then run this class file,
just as if it were compiled from Java source.
The script {\tt CheckR\_6.sh} is based on this idea:
\begin{enumerate}
  \item
  It first runs your compiler with mode {\tt -\flowgen{}} on the C source code.
  If the instructor solution generates an error message,
  then the script checks that your compiler generated an error message
  on the same line.

  \item
  Otherwise, the script runs the assembler on your compiler's output.

  \item
  The script runs the resulting {\tt .class} file on a JVM,
  with one or more input files
  (in case the C source calls {\tt getchar()})
  and checks the output.
\end{enumerate}
As with part~\codegen,
you will need {\tt libc.class},
which implements methods {\tt putchar()} and {\tt getchar()},
to be present in the same directory.

You can make your own test files,
and generate the expected output files
obtained by an executable built with {\tt gcc},
using the {\tt -G} switch.

%======================================================================
\section{Grading} \label{SEC:grading}
%======================================================================

For all students: implement as many or as few features listed below as you wish,
but keep in mind that some features will make testing your code \emph{much}
easier, and a deficit of points will impact your overall grade.
Excess points will count as extra credit.

For code generation ``without short circuiting'',
your compilers will be tested using integer variables
for the condition.
The basic tests for each construct are as follows,
where ${\tt x}$ is an integer variable.
\begin{lstlisting}[numbers=none]
  if (x) { /* statements */ }

  if (x) { /* statements */ } else { /* statements */ }

  while (x) { /* statements */ }

  do { /* statements */ } while (x);

  for (/* initialize */; x; /* update */) { /* statements */ }
\end{lstlisting}
More advanced tests will use an integer expression,
or a single comparison as the condition.

Code generation ``with short circuiting''
will be tested with a variety of conditions
(comparisons and numeric values)
connected with operators \verb|&&|, \verb+||+, and \verb|!|.


\noindent
\begin{gradetable}
  \mainitem{15}{Documentation}
  \\[1mm]
  \inneritem{3}{\tt README.txt}
  \\[1mm]
  \innerpara{%
    How to build the compiler and documentation.
    Updated to show which part \flowgen{} features are implemented.
  }
  \\[1mm]
  \inneritem{12}{\tt developers.pdf}
  \\[1mm]
  \innerpara{%
    New section for part \flowgen{}, that explains
    the purpose of each source file,
    the main data structures used (or how they were updated),
    and gives a high-level overview of how the target code
    is generated.
  }
  \\[4mm]

  \mainitem{10}{Ease of grading}
  \\[1mm]
  \inneritem{4}{Building}
  \\[1mm]
  \innerpara{%
    How easy was it for the graders to build your compiler and
    documentation?
    For full credit, simply running ``{\tt make}''
    should build both the documentation and the compiler executable,
    and running ``{\tt make clean}'' should remove
    all generated files.
  }
  \\[1mm]
  \inneritem{6}{Output and formatting}
  \\[1mm]
  \innerpara{%
    Does the {\tt -o} switch work?
    Is your output formatted correctly?
    Are other messages written to standard error?
  }
  \\[4mm]

  \mainitem{10}{Still works in modes 0 through \codegen}
  \\[4mm]

  \mainitem{15}{Expressions and function calls}
  \\[1mm]
  \mainpara{%
    This includes the most basic functionality from the previous part of the
    project, namely function calls and assignments to variables,
    that will be necessary to test this part of the project.
  }
  \\[4mm]

  \mainitem{5}{Type checking}
  \\[1mm]
  \mainpara{%
    Check that any expressions used as conditions in if statments
    and while, do-while, or for loops
    have a numeric type (char, or int, or float).
    Generate an appropriate error message otherwise.
  }
  \\[4mm]

  \mainitem{55}{Without short circuiting}
  \\[1mm]
  \inneritem{5}{if--then}
  \\[1mm]
  \inneritem{5}{if--then--else}
  \\[1mm]
  \inneritem{5}{while}
  \\[1mm]
  \inneritem{5}{do--while}
  \\[1mm]
  \inneritem{8}{for}
  \\[1mm]
  \inneritem{5}{ternary operator {\tt ?:}}
  \\[1mm]
  \inneritem{5}{{\tt break} (requires a working loop)}
  \\[1mm]
  \inneritem{5}{{\tt continue} (requires a working loop)}
  \\[1mm]
  \inneritem{12}{comparisons: {\tt ==}, {\tt !=}, {\tt >}, {\tt >=}, {\tt <}, {\tt <=} }
  \\[4mm]

  \mainitem{30}{With short circuiting}
  \\[1mm]
  \inneritem{5}{and, or, not}
  \\[1mm]
  \inneritem{5}{comparisons: {\tt ==}, {\tt !=}, {\tt >}, {\tt >=}, {\tt <}, {\tt <=} }
  \\[1mm]
  \inneritem{5}{Boolean assignments}
  \\[1mm]
  \inneritem{5}{if--then, if--then--else, ternary operator}
  \\[1mm]
  \inneritem{5}{while, do--while}
  \\[1mm]
  \inneritem{5}{for}
  \\[4mm]

  \gradeline
  \mainitem{100}{Total for students in 440 (max points is 120)}
  \\
  \mainitem{120}{Total for students in 540 (max points is 140)}
\end{gradetable}


%======================================================================
\section{Submission}
%======================================================================

\begin{table}[h]
\centering

  \begin{tabular}{lcr}
    {\bf Part} & ~~~~ & {\bf Penalty applied} \\ \cline{1-1}\cline{3-3}
    Part 0 && 50\% off \\
    Part 1 && 40\% off \\
    Part \parser && 30\% off \\
    Part \typecheck && 20\% off \\
    Part \codegen && 10\% off
  \end{tabular}

\caption{Penalty applied when re-grading}
\label{TAB:penalties}
\end{table}

Be sure to commit your source code and documentation to your
git repository, and to upload (push) those commits to the server
so that we may grade them.
In Canvas,
indicate which parts you would like us to re-grade for reduced credit
(see Table~\ref{TAB:penalties} for penalty information).
Otherwise, we will grade only part \flowgen{}.

%======================================================================
\section{Examples} \label{SEC:examples}
%======================================================================

\subsection{Input: {\tt hello.c}}

\lstinputlisting{hello.c}

\subsection{Output: {\tt hello.j}}

\lstinputlisting[style=jvm]{hello.j.instr}

\subsection{Input: {\tt fib.c}}

\lstinputlisting{fib.c}

\subsection{Output: {\tt fib.j}}

\lstinputlisting[style=jvm]{fib.j.instr}

\subsection{Input: {\tt short.c}}

\lstinputlisting{short.c}

\subsection{Output: {\tt short.j}}

\lstinputlisting[style=jvm]{short.j.instr}

\subsection{Input: {\tt break1.c}}

\lstinputlisting{break1.c}

\subsection{Output: {\tt break1.j}}

\lstinputlisting[style=jvm]{break1.j.instr}

\subsection{Input: {\tt break2.c}}

\lstinputlisting{break2.c}

\subsection{Error(s) for {\tt break2.c}}

\lstinputlisting[keywordstyle={},numbers=none]{break2.error}

\end{document}
